\documentclass{article}
\usepackage[utf8]{inputenc}
\usepackage[a4paper, total={6in, 8in}]{geometry}
\usepackage{amsmath, amsfonts}
\usepackage{hyperref, graphicx}
\usepackage{tikz}

\DeclareMathOperator*{\argmin}{arg\,min}
\DeclareMathOperator*{\argmax}{arg\,max}

\title{CSCI 145 Problem Set 7}
\author{} % TODO: Put your name here
\date{\today}

\begin{document}

\maketitle

\subsection*{Submission Instructions}

Please upload \textit{your} work by
\textbf{11:59pm Monday October 13, 2025.}
\begin{itemize}
\item You are encouraged to discuss ideas
and work with your classmates. However, you
\textbf{must acknowledge} your collaborators
at the top of each solution on which
you collaborated with others 
and you \textbf{must write} your solutions
independently.
\item Your solutions to theory questions must
be written legibly, or typeset in LaTeX or markdown.
If you would like to use LaTeX, you can import the source of this document (available from the course webpage) to Overleaf.
\item I recommend that you write your solutions to coding questions in a Jupyter notebook using Google Colab.
\item You should submit your solutions as a \textbf{single PDF} via the assignment on Gradescope.
\end{itemize}

\noindent
\textbf{Grading:} The point of the problem set is for \textit{you} to learn. To this end, I hope to disincentivize the use of LLMs by \textbf{not} grading your work for correctness. Instead, you will grade your own work by comparing it to my solutions. This self-grade is due the Friday \textit{after} the problem set is due, also on Gradescope.

\newpage
\section*{Problem 1: Using and Interpreting CNNs}

In this problem, we will explore convolutional neural networks (CNNs) and an interpetability tool called class activation maps (CAM).
\href{https://ezyang.github.io/convolution-visualizer/}{This} tool from class may be helpful for visualizations.

\subsection*{Part A: Kernels by Hand}
Write down one $3 \times 3$ kernel for blurring, one $3 \times 3$ kernel for detecting \textit{horizontal} edges, and one $3 \times 3$ kernel for detecting \textit{diagonal} edges.
Confirm your kernels work as intended on example $2D$ matrices.

\subsection*{Part B: Pretrained CNN}
Using \texttt{torch}, load a ResNet50 architecture (example code \href{https://pytorch.org/hub/nvidia_deeplearningexamples_resnet50/}{here}).
Preprocess an animal photo of your choice, pass it to the model, and inspect the predictions made by the model.
In particular, print the five largest probabilities (apply softmax to the logits), and the corresponding human-readable labels.

\subsection*{Part C: Explaining Predictions via CAM}

\paragraph{Setup.}
Let the last convolutional activation tensor be
\[
\mathbf{A}\in\mathbb{R}^{B\times C\times H\times W},
\]
where $B$ is batch size, $C$ the number of channels, and $H\times W$ the spatial resolution.

Apply \emph{global adaptive average pooling} (GAP) over the spatial dimensions to obtain
\[
\mathbf{z}=\mathrm{GAP}(\mathbf{A})\in\mathbb{R}^{B\times C},\qquad
z_{b,c}=\frac{1}{HW}\sum_{x=1}^{H}\sum_{y=1}^{W} A_{b,c,x,y}.
\]

The final fully connected (linear) classifier has weights
\[
\mathbf{W}\in\mathbb{R}^{K\times C},\quad \mathbf{b}\in\mathbb{R}^{K},
\]
with $K$ classes and row-vectors $\mathbf{w}_k^\top$ ($k=1,\dots,K$). The class logits are
\[
\mathbf{s}=\mathbf{z}\,\mathbf{W}^\top+\mathbf{b}\in\mathbb{R}^{B\times K},\qquad
s_{b,k}=\sum_{c=1}^{C} w_{k,c}\, z_{b,c}+b_k.
\]

\paragraph{Class Activation Map (CAM).}
For class $k$, the CAM before pooling is defined (per sample $b$) as
\[
M_{b,k}(x,y)=\sum_{c=1}^{C} w_{k,c}\, A_{b,c,x,y},
\]
so that $\mathbf{M}_k\in\mathbb{R}^{B\times H\times W}$. For the \emph{predicted} class of sample $b$,
\[
k_b^\ast=\arg\max_{k} s_{b,k},\qquad
M_{b,k_b^\ast}(x,y)=\sum_{c=1}^{C} w_{k_b^\ast,c}\, A_{b,c,x,y}.
\]

\paragraph{Procedure (what to implement).}
\begin{enumerate}
  \item Run a forward pass on the input image(s) and extract $\mathbf{A}$ (e.g., via a forward hook at the last conv layer).
  \item Compute logits $\mathbf{s}$ and, for each sample $b$, find $k_b^\ast=\arg\max_k s_{b,k}$.
  \item For each sample $b$, form the CAM using the corresponding classifier weights:
  \[
  \mathbf{M}_{b} = \sum_{c=1}^{C} w_{k_b^\ast,c}\,\mathbf{A}_{b,c,\cdot,\cdot}\;\in\;\mathbb{R}^{H\times W}.
  \]
  \item Upsample $\mathbf{M}_{b}$ to the input image resolution $(H_0,W_0)$ using \texttt{torch.nn.functional.interpolate} (e.g., \texttt{mode='bilinear', align\_corners=False}):
  \[
  \tilde{\mathbf{M}}_{b}=\mathrm{Interp}\!\left(\mathbf{M}_{b};\,\text{size}=(H_0,W_0)\right).
  \]
  \item Normalize $\tilde{\mathbf{M}}_{b}$ to $[0,1]$ (e.g., min--max) and overlay it as a heatmap on the corresponding input image.
\end{enumerate}


After overlaying the CAM on the input image, what regions appear most responsible for the model's prediction? What do you notice?

%\input{solutions/solution7_1}

\newpage
\section*{Problem 2: Transformers}

\subsection*{Part A: Self-attention by Hand}
You are given a sequence of $T=3$ tokens with $d_{\text{model}}=2$:
\[
\mathbf{X}=
\begin{bmatrix}
1 & 0\\
0 & 1\\
1 & 1
\end{bmatrix}.
\]
Let $d_k=d_v=2$ and
\[
\mathbf{W}_Q=\mathbf{W}_K=\mathbf{W}_V=\mathbf{I}_2.
\]
\begin{enumerate}
  \item Compute $\mathbf{Q}=\mathbf{X}\mathbf{W}_Q,\ \mathbf{K}=\mathbf{X}\mathbf{W}_K,\ \mathbf{V}=\mathbf{X}\mathbf{W}_V$.
  \item Form the unnormalized score matrix $\mathbf{S}=\mathbf{Q}\mathbf{K}^\top/\sqrt{2}$.
  \item Apply a \emph{causal mask} (token $t$ may only attend to $\{1,\dots,t\}$; set masked entries to $-\infty$).
  \item Apply \texttt{softmax} row-wise to obtain attention weights $\mathbf{A}$.
  \item Compute outputs $\mathbf{Y}=\mathbf{A}\mathbf{V}$.
\end{enumerate}
Report $\mathbf{S}$ (masked), $\mathbf{A}$, and $\mathbf{Y}$.

\subsection*{Part B: Self-attention Activations}

Consider a pretrained transformer model.
Select a text input of your choice.
Plot the self-attention weights at several layers (both early and later) in your transformer model.
What patterns do you notice?

\subsection*{Part C: Cross-attention Activations}

Consider a pretrained transformer model for translation.
Select a text input of your choice and translate it.
Now feed the original and translated text to a translation model and plot the cross-attention weights at several layers (both early and later). 
What patterns do you notice?

%\input{solutions/solution7_2}

\end{document}