\documentclass{article}
\usepackage[utf8]{inputenc}
\usepackage[a4paper, total={6in, 8in}]{geometry}
\usepackage{amsmath, amsfonts}
\usepackage{hyperref,bbm, graphicx}
\usepackage{bm}

\DeclareMathOperator*{\argmin}{arg\,min}

\title{CSCI 1051 Problem Set 4}
\author{} % TODO: Put your name here
\date{\today}

\begin{document}

\maketitle

\subsection*{Submission Instructions}

Please upload your solutions by
\textbf{5pm Wednesday January 29, 2025.}
\begin{itemize}
\item You are encouraged to discuss ideas
and work with your classmates. However, you
\textbf{must acknowledge} your collaborators
at the top of each solution on which
you collaborated with others 
and you \textbf{must write} your solutions
independently.
\item Your solutions to theory questions must
be written legibly, or typeset in LaTeX or markdown.
If you would like to use LaTeX, you can import the source of this document 
\href{https://www.rtealwitter.com/deeplearning/psets/pset2.tex}{here}
to Overleaf.
\item I recommend that you write your solutions to coding questions in a Jupyter notebook using Google Colab.
\item You should submit your solutions as a \textbf{single PDF} via the assignment on Gradescope.
\end{itemize}

\newpage
\section*{Problem 1: Interpretability}

\subsection*{Part A: Data, Training, SHAP}

Using \texttt{shap}, load (a subset of) the California dataset.
Using \texttt{sklearn}, train a linear regression and neural network model on the data.
Using \texttt{shap}, apply an explainer of your choice to each model and the dataset.

\subsection*{Part B: Waterfall Plot}

For the same observation in the dataset, make a waterfall plot with the Shapley values for both models. What do you notice?

\subsection*{Part C: Beeswarm Plot}
Make a beeswarm plot with the Shapley values for both models. What do you notice?

\end{document}